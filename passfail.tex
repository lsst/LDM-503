\section{Pass/Fail Criteria}

A test case will be considered ``Passed'' when:
\begin{itemize_single}
\item All of the test steps of the Test Case are completed and
\item All open SPRs from this Test Case are considered noncritical by DMCCB.
\end{itemize_single}

% What is the meaning of "partially passed"? Does a partially passed test case
% count towards % verification of a requirement? Does it mean the test has to
% be re-run? -- JDS, 2017-07-02.
A test case will be considered ``Partially Passed'' when:
\begin{itemize_single}
\item Only a subset of all of the test steps in the Test Case are completed and/or there remain open SPRs which are regarded as critical by the DMCCB, but
\item The overall purpose of the test has been met.
\end{itemize_single}

A test case will be considered ``Failed'' when:
\begin{itemize_single}
\item Only a subset of all of the test steps in the Test Case are completed and/or there remain open SPRs which are regarded as critical by the DMCCB, and
\item The overall purpose of the test has not been met.
\end{itemize_single}

Note that in \citeds{LPM-17} science requirements are described as having a minimum specification, a design specification and a stretch goal.
We preserve these distinctions where they have been made in, for example, the verification framework and automated metric harness.
However for the purposes of pass/fail criteria, it is the design specification that is verified as having been met for a test to pass without intervention of the Software Review Board.

In the event that a requirement is failing its design specification and the minimum specification is invoked, this is an LSST project level issue and is escalated beyond the scope of this plan.
