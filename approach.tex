\section{DM Verification Approach \label{sect:approach}}

Our approach towards verifying the \product\ requirements follows
standard engineering practice.  Each high level component will have at
least one test specification defining a set of tests related to the
design requirements for the component.  These specifications are
represented on the top of \figref{fig:doctree}. Any given requirement
may have several tests associated with it in the specification; these
tests may be phased to account for incremental delivery depending on
the need for certain functionality at a specific time.

The test spec will cover all aspects of the test as outlined in
\secref{sect:tsform}. These high level test specifications may call
out individual lower level test specifications where it makes sense
(either technically or programmatically) to test lower-level components
in isolation.

\subsection{Reports}

As we execute tests we will generate test reports on the pass/fail status of the individual tests related to specific requirements.
This information will allow us to build a Verification Control Document (VCD) (shown at the right of \figref{fig:doctree}).
The VCD will provide the fractional verification status of each DM requirement.
These will also be rolled up to the (higher) level of OSS (Observatory System Specifications; \citeds{LSE-30}) requirements.
\figref {fig:doctree} currently calls for a report from each test spec.
This report may be captured directly in e.g.\ JIRA: it does not necessarily correspond to a separate (e.g. Word or LaTeX) document.

In cases of reports that are generated via automatic (continuous) verification, the report may be in the format of a Jupyter Notebook that simultaneously can serve as test specification and test report and, in some cases, the test script itself.
This is the preferred method, provided the notebook-as-report is satisfactorily captured in DocuShare.

\begin{figure}
\begin{center}
 \includegraphics[angle=-90,width=0.7\textwidth]{images/DocTree}
 \caption{Documentation tree for DM software relating the high level documents to each other. (from \citeds{LDM-294})\label{fig:doctree}}

 \end{center}
 \end{figure}


\begin{figure}[htbp]
	\begin{center}
		\includegraphics[width=0.8\textwidth]{images/DMSDeployment}
		\caption{DM components as deployed during Operations. Where components are
			deployed in multiple locations, the connections between them are labeled with
			the relevant communication protocols. Science payloads are shown in blue.
            For details, refer to \citeds{LDM-148}.
		\label{fig:dmsdeploy}}
	\end{center}
\end{figure}

\subsection{Components Under Test}

\begin{table}
	\caption{Components from LDM-148 with the test specifications to verify them. \label{tab:testspecs}}
	\begin{longtable}{p{0.4\textwidth}p{0.2\textwidth}p{0.2\textwidth}}

\caption{Components of the \product{} system with the test specifications to verify them.
A cyan background indicates that a test specification is currently available; yellow, that one is being drafted at time of writing; orange, that the existing test specification is under revision.
\label{tab:testspecs}} \\

\toprule
                                       & \multicolumn{2}{c}{\textbf{Specification}} \\
\multicolumn{1}{c}{\textbf{Component}} & \multicolumn{1}{c}{\textbf{Requirement}} & \multicolumn{1}{c}{\textbf{Test}} \\
\toprule
Data Backbone                     & \cellcolor{dmorange} LDM-635 & \cellcolor{dmyellow} LDM-535 \\
LSP Services                      &   \cellcolor{dmblue} LDM-554 &   \cellcolor{dmblue} LDM-540 \\
Alert Distribution service        & \cellcolor{dmyellow} LDM-638 &                              \\
Archiving service                 & \cellcolor{dmyellow} LDM-638 &   \cellcolor{dmblue} LDM-538 \\
Prompt Processing service         & \cellcolor{dmyellow} LDM-638 &   \cellcolor{dmblue} LDM-533 \\
Batch Production Software         &                              &                              \\
Alert Distribution Software       &                              &   \cellcolor{dmblue} LDM-533 \\
EFD Transformation Software       &                              &                              \\
Header Service Software           &                              &                              \\
Alert Production Software         &   \cellcolor{dmblue} LDM-602 &   \cellcolor{dmblue} LDM-533 \\
Calibration Software              &                              &                              \\
DR Production Software            &    \cellcolor{dmblue} LDM562 &   \cellcolor{dmblue} LDM-534 \\
Data Butler                       &                              &                              \\
Distributed Database (Qserv)      &                              &   \cellcolor{dmblue} LDM-552 \\
Commissioning Cluster Enclave     &                              & \cellcolor{dmyellow} LDM-541 \\
DAC Enclaves                      &                              & \cellcolor{dmyellow} LDM-539 \\
Offline Production Enclave (NCSA) &                              & \cellcolor{dmyellow} LDM-532 \\
Prompt Enclaves                   &                              &                              \\
Data Management Acceptance        &    \cellcolor{dmblue} LSE-61 & \cellcolor{dmorange} LDM-639 \\
\bottomrule

\end{longtable}

\end{table}

The components of the DM ystem are outlined in \citeds{LDM-294} and detailed in \citeds{LDM-148}. At a high level these components are represented in \figref{fig:dmsdeploy}. Based on those components we can see the set of Test Specifications needed in \tabref{tab:testspecs}. At time of writing, document numbers are not available for all second-level components.

The test items covered in this test plan are:

\begin{itemize_single}
\item \product\ and its primary components for testing and integration purposes. These are listed in Table \ref{tab:testspecs}. All components listed in orange and yellow have specifications in the corresponding documents listed. Major sub-components in white may have individual test specifications or be addressed in the component they are under depending on applicable factors such as whether they are scheduled for testing at the same time and/or whether they share architectural components or are largely distinct.

\item The external interfaces between \product{} and other sub-systems. These are described in \href{https://ls.st/Collection-5201}{DocuShare collection 5201}.

\item Operational procedures like Data Release Process, the Software Release Process and the Security Plan.

\end{itemize_single}

\subsection{Testing Specification Document Format}\label{sect:tsform}

The testing specification documents will be drawn up in conjunction with the LSST Systems Engineer. In all cases they will include:

\begin{itemize_single}

\item A list of components being tested within the scope of the test specification document.

\item A list of features in those components that are being explicitly tested.

\item The relationship between features under test and the identified requirements for the component.

\item A description of the environment in which the tests are carried out (e.g.\ hardware platform) and a description of how they differ from the operational system in tests prior to final integration (e.g.\ interfaces that may be mocked without affecting that component's testing).

\item The inputs (such as data, API load, etc.) that are to be used in the test.

\item Pass-fail criteria on any metrics or other measurements.

\item How any outputs that are used to determine pass/fail (e.g.\ data or metrics) are to be published or otherwise made available.

\item A software quality assurance manifest, listing (as relevant) code repositories, configuration information, release/distribution methods and applicable documentation (such as installation instructions, developer guide, user guide etc.)

\end{itemize_single}
