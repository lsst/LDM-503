\section{Verification Tests \label{sect:approach}}

Our approach towards verifying the successful delivery of the \product{} System follows standard engineering practice.

We regard the system as being successfully completed when all of the high level requirements placed upon it, as defined in \citeds{LSE-61} --- the \emph{Data Management System Requirements} --- have been verified.
The approach which will be taken to verifying each requirement is described in \citeds{LDM-639}, the \emph{DM Acceptance Test Specification}.
This test specification covers all aspects of the tests, as described in \secref{sect:tsform}.
Any given requirement may have multiple test cases associated with it in the specification, and these tests will be phased to account for incremental delivery depending on the need for certain functionality at a specific time.

In addition to the high level requirements on the overall \product{} system, lower-level requirements documents describe requirements placed upon specific parts of the system (for example, \citeds{LDM-554} provides requirements on the LSST Science Platform, and \citeds{LDM-555} on the DM database system).
Each of these requirements documents is accompanied by a test specification (\citeds{LDM-540} and \citeds{LDM-552} in the case of the previous examples; see also Table \ref{tab:testspecs}), with the relationship between them being the same as for the high-level requirements.

In summer 2018, a major update of the \product{} product tree has been performed. The new lower-level products will be covered by requirements and test documents as per need bases, and depending on the resources available.

In general, in order to avoid duplication of test definition and execution, when a requirement is flowed down to lower-level requirements, and therefore tested at lower level, the high level test activity may be just an inspection.
Detailed high level test cases are required, in case that a LSE-61 requirement is not decomposed in lower-level product requirements.
Detailed high level test cases are required if an  LSE-61 requirement is not decomposed in lower-level product requirements.

Although individual test cases may be executed at any time, it is anticipated that major testing campaigns will be undertaken to demonstrate the successful completion of major milestones in the \product{} construction effort.
The schedule for these milestones is shown in \secref{sect:schedule}, while \secref{sect:milestones} provides further details as to the contents of each one.
In addition, each low level product owner, can define specific test campaign to verify the coverage of the lover level requirements.
An example could be a specific software release made for a specific purpose (start of commissioning).

As explained at the beginning of this section, the validation of the LDM-61 requirements will prove that the \product{} requirements have been properly implemented, therefor this will be done at the end of the test activity.
LDM-61 requirements have a specific priority assigned. Each of them will be verified depending on the priority assigned.

\subsection{Managing and Reporting Test Execution}
\label{sect:reports}

As described above, requirements and test specifications are provided in baselined documents.
These documents provide curated views on the Jira \emph{LSST Verification and Validation} project which underlies the LSST-wide test effort.
The Jira system provides ``test scripts'' that testers will follow when carrying out tests, and tracks information about test execution and results.
This information enables us to generate reports on the execution of each test\footnote{These test reports may, on occasion, be issued as Word or LaTeX documents, but this is not, in general, required.}, and ultimately to build a Verification Control Document (VCD; see \figref{fig:doctree}).
The VCD will provide the verification status of each DM requirement (in terms of the fraction of test cases pertaining to that requirement which have been successfully executed).


The \figref{fig:dmvaldoctree} highlights the different DM validation documents types and how they related each other.

\begin{figure}[htbp]
        \begin{center}
                \includegraphics[width=0.9\textwidth]{images/DMValidationDocTree}
                \caption{DM validation document tree overview.}
                \label{fig:dmvaldoctree}
        \end{center}
\end{figure}



\begin{figure}
\begin{center}
 \includegraphics[angle=-90,width=0.65\textwidth]{images/DocTree}

 \caption{The \product{} document tree.}
 \label{fig:doctree}

\end{center}
\end{figure}

\begin{figure}[htbp]
	\begin{center}
		\includegraphics[width=0.9\textwidth, trim={0cm 15cm 0cm 0cm}]{images/DMSDeployment}
		\caption{DM components as deployed during Operations. For details, refer to \citeds{LDM-148}.
		\label{fig:dmsdeploy}}
	\end{center}
\end{figure}

\subsection{Components Under Test}\label{sect:components}

The components of the DM system are outlined in \citeds{LDM-294} and described in detail in \citeds{LDM-148}; a summary is shown in \figref{fig:dmsdeploy}.
The test specifications covering these components are shown in \tabref{tab:testspecs}, but note that, at time of writing, the document tree is being refactored and document numbers are not currently available for all components (refactored as expected).
Based on those components we can see the set of Test Specifications needed in \tabref{tab:testspecs}.

The test items covered in this test plan are:

\begin{itemize_single}

\item The \product{} System and its primary components for testing and integration purposes. These are listed in Table \ref{tab:testspecs}.
All components listed in orange and yellow have specifications in the corresponding documents listed.
Major sub-components in white may have individual test specifications or be addressed in the component they are under depending on applicable factors such as whether they are scheduled for testing at the same time and/or whether they share architectural components or are largely distinct.
Other components from the product tree may be covered on best effort.

\item The external interfaces between \product{} and other sub-systems. These are described in \href{https://ls.st/Collection-5201}{DocuShare collection 5201}. These verification activities will concur to the global coverage of LSST requirements, but will not be included in the calculation of the DM verification control document.


\item Operational procedures like Data Release Process  are mainly covered by test milestones. These include  multiple Operation rehearsals have been planned.

\end{itemize_single}

\begin{longtable}{p{0.4\textwidth}|p{0.2\textwidth}}
\multicolumn{1}{c}{\textbf{Component}} & \multicolumn{1}{c}{\textbf{Test Specification}} \\ \hline

\rowcolor{yellow!50!white} NCSA Enclave &                 LDM-532 \\ \hline
\rowcolor{cyan!50!white}   Level 1 System   &             LDM-533 \\ \hline
\rowcolor{cyan!50!white}   Level 2 System   &             LDM-534 \\ \hline
\rowcolor{yellow!50!white} Data Backbone &                LDM-535 \\ \hline
\rowcolor{yellow!50!white} Data Services &                LDM-536 \\ \hline
\rowcolor{cyan!50!white}   Data Backbone Infrastructure & LDM-537 \\ \hline
\rowcolor{cyan!50!white}   Raw Image Archiving Service &  LDM-538 \\ \hline
\rowcolor{cyan!50!white}   Science Platform &             LDM-540 \\ \hline
\rowcolor{yellow!50!white} Commisioning Cluster &         LDM-541 \\ \hline
\rowcolor{cyan!50!white}   Qserv &                        LDM-552 \\ \hline
\rowcolor{orange!50!white} \product{} Acceptance &        LDM-639 \\ \hline

\end{longtable}


Following the test approach here described, the milestone consist in a Jira test plan and one or more test cycles.
The test cases to be executed in a milestone, and listed in the test cycle, may come from different component's test specification.
Therefore, for each milestone described in \secref{sect:schedule} and \secref{sect:milestones}, shall be specified which components are involved.

%\subsection{Testing Specification Document Format}\label{sect:tsform}
\subsection{Test Approach Overview}\label{sect:tsform}

This section gives an overview of the approach, facilities and documents involved in the verification process.


\subsubsection{Tools}

{\bf MagicDraw}: is the tool where all modeling and requirements definition is done. System Engineering will derive the final coverage of the LSST requirement using MagicDraw, therefore the importance to have all information collected here.

{\bf Jira and Test Management Plugin}: Jira is the issue tracking system. The Test Management plugin (ATM) add the capability to handle verification activities in Jira.

{\bf Sinedia}: this is the synchronization tool between MagicDraw and Jira.

{\bf Extraction Scripts}: these scripts will permit to extract from Jira the relevant test documentation: the test specifications and the test plan and report.


\subsubsection{Requirements and Test Objects}

{\bf Requirements}: are defined in MagicDraw and the change control process is applied to them. High level requirements, like the one documented in LSE-61, are managed by the project CCB, the low level requirements are change controlled by the DM CCB.
Requirements are collected in Requirements Specification documents and  are not synchronized in Jira.

{\bf Verification Elements}: each requirement is decomposed in one or more verification elements. A verification element is an aspect of the requirement that can be tested, without considering the full requirement.
The verification elements are created in MagicDraw and synchronized with Jira.
Product owners must update them in Jira with a proper description and other characterization parameters.
System Engineering will synchronize them with MagicDraw once the update is complete.
Verification elements in Jira are implemented with a specific issue type, and are normal Jira issues.

{\bf Test Cases}: are the definition of test procedure to be executed in order to prove that the related verification elements are properly tested. A test case can be related with many verification elements. It is not a Jira issue, but a test object provided by ATM.

{\bf Test Plans}: are the definition of the test activity to be executed  at a specific moment in time in order to fulfill a specific purpose, for example validate a release or complete a verification milestone.
The test plan is just the definition of the activity, and will not contain nor relate directly to any test case.
A Test Plan can be related to multiple Test Cycles and the test cycles will be related to the test cases.
It is not a Jira issue, but a test object provided by ATM.

{\bf Test Cycles}: orchestrate the execution of the test. In the documentation are called sometime Test Runs.
They are Jira objects that include multiple test cases to be executed in a specific order and configuration.
Test Cycles are related to Test Plans, and contains list of test cases.
A test case can be included only once in a test cycle. If a test case need to be executed with two different configuration during a test activity, two different test cycles need to be used.
It is not a Jira issue, but a test object provided by ATM.

{\bf Deviations}: in the case that it is not possible to successfully verify a Verification element, due to a test failure, a request for deviation can be opened.
This is implemented trough Deviation Jira issue types.

For more details on the different Jira objects and how to use them, refer to the following confluence page
\url{https://confluence.lsstcorp.org/display/DM/DM+Test+Approach}.

Workflows on the different objects are available in SE documentation.


\subsubsection{Test Documents}

Though all tests information is contained in MagicDraw and Jira, it is important to have a  baselined test documentation in Docushare. .
%This is already happening for the requirements that are captured in requirements documents, and when approved issued into Docushare.

The {\bf Test Specification} document consists of a collections of test cases that cover a specific \product{} product as defined in the document tree and in \secref{sect:components}.
These collections of test cases are available in in Jira as views in the ``LSST Verification and Validation'' project, and are organized in folders.

The test specification sections are organized as follows:

\begin{itemize}
\item Introduction: includes subsections ``Objective'' and ``Scope'' to edit and maintain manually by the author, not in Jira, and upload changes it in github.
\item Approach: to edit and maintain manually by the author, not in Jira, and upload changes it in github. The relevant subsections are: ``Tasks and criteria'', `` Features to be tested'', ``Features not to be tested'', ``Pass/fail criteria'', ``Suspension criteria'', ``Naming convention''.
\item Test Case Summary: this is a summary table extracted directly from Jira that will summarize all test cases included in the test specification. Do not edit it manually, all changes need to be done in Jira.
\item Test Case Specification: this section will include all test cases details extracted automatically from Jira. Do not edit it manually, al changes need to be done in Jira.
\end{itemize}

One additional section or an appendix should be added to provide pointers to documentation  on the verification elements.

Additional sections can be added in appendix.

The test specification is a living document, test cases may be added or updated in Jira and new issues of the same document will be generated, approved and uploaded in Docushare.

Note: never update a test specification in section 3 or 4, since this will be overwritten by the extraction scripts. Note also that the old naming convention is not more applicable, since the Test Case identification is assigned automatically by Jira. Existing test cases will preserve the old identification in the subject.

Taking advantage of the continuous integration process executed in Travis, the {\bf extraction scripts} will be integration in the build of the pdf in order to extract dynamically the section 3 and 4 content from Jira, on a regular basis, to be defined.


The {\bf Test Plan and Report} is the documented that will include all information related to a test campaign.
This includes all information from the test plan and the related test runs.
Do not edit it directly, all change need to be done in Jira.

The author need to define the structure of the document and upload it in github. The continuous integration in Travis will take care to build the document on a regular bases, to be defined, downloading the corresponding information from Jira.

The test plan and report scope is constrained on a single major test activity. It may only be updated, generating a new issues of the document, in case of re-execution of a subset of tests after the test campaign is concluded and the document is already in Docushare.
In general a new test activity, due to a new milestone or release to be tested, imply a creation of a new test plan and report.

Note that in the past this document was identified as "Test Report" but is now
the "Test Plan and Report" in order to better identify the two types of information it contains: planning of the test activity and report on the executed test cases. The document type acronym will remain DMTR.

Note also that, the reports considered in this context are only the documents which scope is to provide information on test campaigns documented and executed using the test management infrastructure in Jira. Release characterization reports are not relevant here, unless their documentation, planning and execution will not follows these guidelines and use the same tooling.


\subsubsection{Approval Procedure}

The approval of the test specification is already addressed via RFC as done for all the DM-CCB controlled documents.

New or updated test cases shall be in status ``draft''. After the test specification has been approved via RFC, their status need to be set to ``approved''.
Only then the document can be regenerated and uploaded in Docushare.

The approval of the test plan and report, is a bit different.
For practical reasons, the test report is completed in 2 steps:

\begin{itemize}
\item before the test execution: the test plan and the test cycle should be created in Jira, in order to demonstrate that the test is ready to start.
\item after the test execution: the document is finalized including the reports and comments on the execution of the test cases, issues raised during the test campaign and the a statement summarizing the global test campaign result.
\end{itemize}

Only the final document is relevant for deriving the global coverage of the DM requirements, but it is important that both steps are approved separately. In this way we ensure all conditions are met to  execute the tests.

The approval of the first step comes after verification,  by the product owner, that all needed information is provided in Jira and therefore in the test plan and report document.
The formality of  issuing  the document in Docushare can be skipped, but a coherent draft has to be available in DMTR-XXX.lsst.io and the status of the corresponding test plan has to be set to "approved".
% In Jira ???
Stakeholders that may have some expectation on the activity results, like scientists, T/CAMs or others, should  be involved before the approval.
% SO we are using RFC ??

The final approval of the document, after the test activity is concluded and all information is collected in Jira, should be explicitly given by whoever requested the test activity to be carried out.

\subsubsection{The DM VCD}

The global project wide verification control document will be obtained from MagicDraw by the System Engineering team.

However, Data Management will be able to provide a DM specific verification control document, that will show the coverage of each system specific requirement, and their relation with test cases and test executions.
This information will be derived only from Jira using the ``extraction scripts'' already used for the generation of the test specification and test plan and report.

An example, at least in structure, of the VCD is given in \tabref{tab:dmvcd}.

% Status
\newcommand{\pass}{\cellcolor{dmgreen} \textbf{Passed}}
\newcommand{\fail}{\cellcolor{dmred} \textbf{Failed}}
\newcommand{\nr}{\cellcolor{dmorange} \textbf{Not run}}

% References
\newcommand{\vcdJiraRef}[1]{\scriptsize{#1}}
\newcommand{\vcdDocRef}[1]{\scriptsize{#1}}

\afterpage{
\begin{landscape}

% Ugly hack for centering on page
\null
\vspace{1.0cm}

% Use a long caption width, then set back to the old value later.
\newlength{\LTcapwidthold}
\setlength{\LTcapwidthold}{\LTcapwidth}
\setlength{\LTcapwidth}{\textheight}

\begin{longtable}{lllllll}

\caption{
An example of the structure of the Verification Control Document.
Identifiers are intented to be illustrative only, and may not correspond to existing documents, requirements, verification elements, test cases, or test plans and reports.
In this example, DMS-REQ-0001 is a requirement which originates within DM (ie, it has no higher level parent.
It is verified directly by test case LVV-T2, and has a child requirement (DMS-LSP-REQ-0002) which is verified by both LVV-T1 and LVV-T4.
DMS-REQ-0002 is a child of OSS-REQ-0007, but itself has no parents.
It contains one verification element, which is, in turn, verified by a two test cases.
}
\label{tab:dmvcd}

\\
\toprule
\textbf{Parent} & \textbf{Requirement} & \textbf{Child} & \textbf{Verification Element} & \textbf{Test Case} & \textbf{Run Date} & \textbf{Test Status} \\
\toprule

                   & DMS-REQ-0001       & DMS-LSP-REQ-0002    & DMS-REQ-0001-V-01  & LVV-T1              & 2018-11-11           & \pass{} \\
                   & \vcdDocRef{LSE-61} & \vcdDocRef{LDM-554} & \vcdJiraRef{LVV-1} & \vcdDocRef{LDM-639} & \vcdDocRef{DMTR-099} & \cellcolor{dmgreen} \\ \cmidrule{5-7}
                   &                    &                     &                    & LVV-T4              & 2018-11-11           & \pass{} \\
                   &                    &                     &                    & \vcdDocRef{LDM-639} & \vcdDocRef{DMTR-099} & \cellcolor{dmgreen} \\ \cmidrule{3-7}
                   &                    &                     & DMS-REQ-0001-V-02  & LVV-T2              &                      & \nr{} \\
                   &                    &                     & \vcdJiraRef{LVV-1} & \vcdDocRef{LDM-535} &                      & \cellcolor{dmorange} \\ \cmidrule{1-7}

OSS-REQ-0007       & DMS-REQ-0002       &                     & DMS-REQ-0002-V-01  & LVV-T3              & 2018-11-11           & \fail{} \\
\vcdDocRef{LSE-30} & \vcdDocRef{LSE-61} &                     & \vcdJiraRef{LVV-3} & \vcdDocRef{LDM-639} & \vcdDocRef{DMTR-099} & \cellcolor{dmred} \\ \cmidrule{5-7}
                   &                    &                     &                    & LVV-T2              &                      & \nr{} \\
                   &                    &                     &                    & \vcdDocRef{LDM-535} &                      & \cellcolor{dmorange} \\

\bottomrule

\end{longtable}
\setlength{\LTcapwidth}{\LTcapwidthold}
\end{landscape}
}


\subsection{Roles and Personnel}
\label{sect:roles}


For each test case is required to specify the ``owner''.
This in general is the \emph{tester} that will be in charge of executing the test case.
However, when assigning a test case to a specific test cycle, it is possible to specify a different name.

The tester is responsible for executing the test cases following the script provided in the Jira ``LSST Verification and Validation'' project (\secref{sect:tsform}).

Testers submit details of test execution to Jira project, where it is used to log test execution and may be used to generate test reports.
The information captured in Jira will also be used to populate the Verification Control Document (see \secref{sect:approach}).

Tests and procedures will sometimes fail: a test case may be re-run several times until it passes, but testers will log an explanation than indicates that any failures were understood (e.g.\ they were due to a fault that was fixed) or repeated sufficient times to ensure that passing the test was not transient success.
Issues which cannot be resolved by the tester in the course of carrying out the test will be reported as ``Software Problem Reports'' (SPRs) through the \product{} ticketing system (the Jira ``Data Management'' project at the time of this document revision).
The DMCCB, or an individual designated by it, will be tasked with assessing the SPRs determining the timescale for re-executing the test procedure.
If a new version of the software is provided, a test cycle can be create in order to document the new run in a separate Jira element, and keep track of the history.

Other parties that have a relevant role in \product\ verification are identified in the appropriate sections of the document; these are involved in their primary capacity (e.g.\ the DM Systems Engineer) and so are not individually listed in this section.

\subsection{Pass/Fail Criteria}

A test case will be considered ``Passed'' when:

\begin{itemize_single}
\item{All of the test steps of the Test Case are completed; and}
\item{All open SPRs from this Test Case are considered noncritical by DMCCB.}
\end{itemize_single}

A test case will be considered ``Partially Passed'' when:

\begin{itemize_single}
\item{Only a subset of all of the test steps in the Test Case are completed and/or there remain open SPRs which are regarded as critical by the DMCCB; but}
\item{The DMCCB regards overall purpose of the test as having been met.}
\end{itemize_single}

A test case will be considered ``Failed'' when:

\begin{itemize_single}
\item{Only a subset of all of the test steps in the Test Case are completed and/or there remain open SPRs which are regarded as critical by the DMCCB; and}
\item{The DMCCB regards overall purpose of the test as not having been met.}
\end{itemize_single}

Note that in \citeds{LPM-17} science requirements are described as having a minimum specification, a design specification and a stretch goal.
While we preserve these distinctions in some DM tooling for internal tracking purposes, for the purposes of these tests, it is the design specification that is verified as having been met for a test to pass without intervention of the DMCCB.
Ultimately, if it proves impossible to satisfy a requirement at design specification, LSST Project level approval is required to accept the minimum specification.

\subsection{Constraints and Limitations}

\subsubsection{Procedural and Technical Limitations}

\begin{itemize}


  \item{The \product{} system must be verified before the complete LSST system can be completed. Verification is therefore carried out using precursor datasets~\footnote{e.g. from Hyper Suprime-Cam; \secref{LDM-503-02}}, simulated data, and --- where available --- with engineering and pre-release data from the as yet incomplete LSST system.}

  \item{Metric measurements and operational rehearsals during construction may not involve critical operational systems that are still in development. For example, while computational performance is being measured, computationally dominant algorithmic steps such as deblending and multi-epoch fitting may only be modeled, since they have not yet been implemented; operational rehearsals are done without the factory LSST workflow system; etc.}

\end{itemize}

\subsubsection{Requirements Traceability Constraints}

The \product{} verification plan is based entirely on requirements captured in the DM System Requirements (\citeds{LSE-61}).
It does not refer to higher level requirements documentation, such as the LSST System Requirements (\citeds{LSE-29}) or the Observatory System Specifications (\citeds{LSE-30}); rather, we assume that all higher level requirements have been correctly flowed down to DM.
In practice, the Systems Engineering team continues to refine the flow-down of higher level requirements and issue updates to \citeds{LSE-61}; this test plan must both anticipate and be responsive to those updates.
