\section{Introduction \label{sect:intro}}

This document provides an introduction to and overview of the approach to verification and validation which bas been adopted by the LSST \product{} Subsystem.
Broadly, this approach consists of three aspects:

\begin{itemize}

  \item{\emph{Verification} that the \product{} system as delivered meets the requirements placed upon it;}
  \item{\emph{Validation} that the system as delivered meets the needs of the scientific community;}
  \item{\emph{Rehearsing} the sustained operation of the system in operational scenarios.}

\end{itemize}

This documentation describes how LSST \product{} is addressing each of these three requirements, and describes a series of high-level milestones and the accompanying test schedule.
In addition, it briefly discusses the software development infrastructure that has been developed to support all three of these aspects of testing.

\subsection{Objectives \label{sect:objectives}}

We describe the test and verification approach for \product{} and describe various constraints and limitations in the testing to be performed.
We also describe the program of rehearsals which will be undertaken to demonstrate the sustained operation of the \product{} system, and the validation exercises which will be performed on the partially and fully integrated system.
We do not describe all tests in detail; those are described in dedicated test specifications for major components of \product{}.

\subsection{Scope \label{sect:scope}}

This provides the approach and plan for all of \product{}.
It covers interfaces between \product{} and components from other LSST subsystems but nothing outside of \product{}.
This document is change-controlled by the DMCCB and will be updated in response to any requirements updates or changes of approach.

\subsection{Assumptions}

We will run large scale verification exercisess in order to demonstrate the system's end-to-end capability against its design specifications.
A large amount of informal science verification and validation will be done in the the teams and documented in technical notes; in this test plan we are looking for verification of the broader system, demonstration of its \emph{operability} --- i.e. whether it can be run every day for the 10 year planned survey with a reasonable level of operational support --  and to validate its capability to meet the scientific expectations of the community.

\subsection{Applicable Documents \label{sect:ad}}

When applicable documents change a change may be required in this document.
\begin{tabbing}
AUTH-NUM\= \kill
\citeds{LPM-55} \>	LSST Quality  Assurance Plan \\
\citeds{LDM-148} \>	DM Architecture\\
\citeds{LDM-294} \>	DM Project Management Plan   \\
\citeds{LDM-639} \>	DM Acceptance Test Specification \\
% perhaps \citell{LL:AUTH-code}\>	Software Requirements Specification for \CU,\\
\end{tabbing}

\subsection{References}

\renewcommand{\refname}{}
\bibliography{lsst,gaia_livelink_valid,refs,books,refs_ads}

\subsection{Definitions, Acronyms, and Abbreviations \label{sect:acronyms}}
\phantom{ } % suppresses a page break before the longtable.
% include acronyms.tex generated by the acronyms.csh (GaiaTools)
\addtocounter{table}{-1}
\begin{longtable}{|l|p{0.8\textwidth}|}\hline 
\textbf{Acronym} & \textbf{Description}  \\\hline
API&Application Programming Interface \\\hline
CI&Configuration Item \\\hline
CPU&Central Processing Unit \\\hline
CTIO&Cerro Tololo Inter-American Observatory \\\hline
DAC&Data Access Center \\\hline
DAX&Data Access Services \\\hline
DBB&Data BackBone \\\hline
DM&Data Management \\\hline
DMCCB&DM Change Control Board \\\hline
DRP&Data Release Production \\\hline
EFD&Engineering Facilities Database \\\hline
EPO&Education and Public Outreach \\\hline
HSC&Hyper Suprime-Cam \\\hline
ICD&Interface Control Document \\\hline
ID&Identifier (Identification) \\\hline
JIRA&issue tracking product (not an acronym, but a truncation of Gojira, the Japanese name for Godzilla) \\\hline
KPM&Key Performance Metric \\\hline
LSST&Large Synoptic Survey Telescope \\\hline
LaTeX&(Leslie) Lamport TeX (document markup language and document preparation system) \\\hline
NCSA&National Center for Supercomputing Applications \\\hline
OCS&Observatory Control System \\\hline
OSS&Operations Support System \\\hline
PMCS&Project Management Control System \\\hline
PSF&Point Spread Function \\\hline
QA&Quality Assurance \\\hline
QC&Quality Control \\\hline
Qserv&Proprietary LSST Database system \\\hline
RC&Reference Catalog \\\hline
SPR&Software Problem Report \\\hline
SQuaRE&Science Quality and Reliability Engineering \\\hline
SST&System Science Team \\\hline
SV&Science Validation \\\hline
TB&TeraByte \\\hline
TBD&To Be Defined (Determined) \\\hline
UX&User interface widget \\\hline
VC&Verification Cluster \\\hline
VCD&Verification Control Document \\\hline
WCS&World Coordinate System \\\hline
\end{longtable}

