


\section{Introduction \label{sect:intro}}

{\bf THERE IS ONLY ONE OF THESE YOU PROBABLY WANT AN STS}


\subsection{Objectives \label{sect:objectives}}

The Software Test Plan describes the system being tested, summarising the system context and decomposition. It sets out the test and verification approach for the system and describes constraints and limitations in the testing to be performed. The STP describes the unit and integration tests for the component modules of the system and describes the validation tests to be performed on the fully integrated system. 

\subsection{Scope \label{sect:scope}}

The Software Test Plan is to be executed by the CU prior to delivery to the DPC where the system will be operated. The DPC will execute integration and acceptance test involving this system within the context of the DPC processing systems.
This document will be updated during the different Gaia cycle phases, according to the requirements updates.

\subsection{Assumptions}  
This paragraph is optional. It describes the preliminary assumptions on which the overall testing strategy are based.
 
\subsection{Applicable Documents \label{sect:ad}}
When applicable documents change a change may be required in this document.
\begin{tabbing}
AUTH-NUM\= \kill 
\citell{LL:TL-001}\>	DPAC Product Assurance Plan \\
\citell{LL:AUTH-XXX} \>	Software Development Plan for \CU \\
\citell{LL:AUTH-XXX}\>	Software Requirements Specification for \product,\\
% perhaps \citell{LL:AUTH-code}\>	Software Requirements Specification for \CU,\\
\end{tabbing}

\subsection{Reference Documents}

\renewcommand{\refname}{}
\bibliographystyle{gaia_aa}
\bibliography{gaia_livelink_valid,gaia_drafts,gaia_refs,gaia_books,gaia_refs_ads}

\subsection{Definitions, acronyms, and abbreviations \label{sect:acronyms}} 
% include acronyms.tex generated by the acronyms.csh (GaiaTools)
The following table has been generated from the on-line Gaia acronym list:
\newline\newline%decrement table counter so table sin doc start at 1.
\addtocounter{table}{-1}
\begin{longtable}{|l|p{0.8\textwidth}|}\hline 
\textbf{Acronym} & \textbf{Description}  \\\hline
CU&Coordination Unit (in DPAC) \\\hline
DPAC&Data Processing and Analysis Consortium \\\hline
DPC&Data Processing Centre \\\hline
OF&Object Feature (source packet) \\\hline
SP&Software Product \\\hline
SPR&Software Problem Report \\\hline
SRS&Software Requirements Specification \\\hline
STP&Software Test Plan \\\hline
STS&Star Tracker System \\\hline
SVN&SubVersioN \\\hline
\end{longtable} 


\section{Test Items}

The test items covered in this test plan are \product \ and its consituent components:

\begin{itemize_single}
\item All the produsct - from KT diagrams

\item Interfaces  
\item Procedures like Data release 
\end{itemize_single}


\section{Roles and Reporting}

Tester report issues through Jira, but what other mechanisms will be used?

WHo directs OPS rehersals .. ?

Reports on rehersals .. issues and 


Handeling failures - time ines for fix. 


\subsection{Pass/Fail Criteria}

The Software Review Board will meet once a full run of all Test Cases has been performed, and subsequently after a complete run of all outstanding Test Cases.

A Test Case will be considered ``Passed'' when:
\begin{itemize_single}
\item All of the test steps of the Test Case are completed and
\item All open SPRs from this Test Case agreed in Software Review Board are considered noncritical.
\end{itemize_single}

A Test Case will be considered ``Partially Passed'' when:
\begin{itemize_single}
\item Only a subset of all of the test steps in the Test Case are completed but the overall purpose of the test has been met and
\item Any critical SPRs from this Test Case agreed in Software Review Board are still not closed.
\end{itemize_single}

A Test Case will be considered ``Failed'' when:
\begin{itemize_single}
\item Only a subset of all of the test steps in the Test Case are completed and the overall purpose of the test has not been met and
\item Any critical SPRs from this Test Case agreed in Software Review Board are still not closed.
\end{itemize_single}

\section{Constraints and Limitations}

Describes the limitations and the constraints which apply to CU level tests of the system. lack of computing resources may mean that datasets are smaller or that full accuracy cannot be achieved. Explain what must be validated in the DPC tests

\section{Master Schedule}

The schedule for testing the system until launch. If some modules are scheduled for development after other, explain dependencies and impact on integration and validation tests.


Nightly Tests \\

Weekly Integration test with data .. \\

Interface tests ( 2by 2 and integrated E2E, Internal and External)) \\

End to End Tests ?? Freeze software for Ops .. \\

WISE data to PDAC - ...\\

HSC reprocessing - yes see the data and also validate the ops platform . Validate some procedures like install some procedures etc .. \\

ZTF Alerts processing to valiate ALerts pipe .. \\

2018 Specrograph Data Acquasiitong Test..

2018 - Ops rehsal for comissioning - with a weeks comissioning say - pick which parts of plan we could reherase.\\

2019 - Ops rehsal \#2 for comissioning - more complete .\\

2020 - Ops Rehersal Data Release (Comisisoning Data)\\
2021 - Ops Rehersal Data Release (Regular Data)\\


\section{Validation Tools}
\subsection{Introduction}

To evaluate the correctness of the generated data and the systems performances a set of tools may be developed or used. These
tools will provide the means to facilitate the validation tasks. 
Following subsections describe the various tools that can used in the \product validation (e.g. data comparison tools, analysis tools, etc).

\subsection{Data Comparison Tools}
This type of test tools are used to manage products in terms of:
\begin{itemize_single}
\item Comparison of a product generated during a test execution w.r.t. the relevant reference product
\item Non regression verification comparing output products generated by different versions of the same system
\item Measurement of quality degradation due to perturbed inputs
\end{itemize_single}
It allows:
\begin{itemize_single}
\item Product analysis
\item Decoding of generated product allowing to read the most significant data of the product itself
\item Visualisation of the values of a single selected field
\item Apply an accuracy to the comparison
\item Comparing specific parts of the products
\item Filtering using flags values
\end{itemize_single}

\subsection{Data Transformation Tools}
These tools allow the data to be transformed to other formatted data.

\subsection{Analysis Tools}
Descriptions of the performance monitoring tools, profilers, test coverage programs... used in the Performance evaluation tests.\\
...

\section{Unit and Integration Tests}

\subsection{Approach}

Unit and Integration Tests will be automatically executed through the JUnit test framework. The descriptions of the test below are extracted from the test cases code and documentation.The results of Unit and Integration Test to be included in the Sofwtare Test Report will be generated automatically from the output of the execution of the tests by JUnit. A script will be provided to perform thes processing steps.

Module identification? (module tag in class header? mapping file?)

\subsection{Test Coverage}

Test coverage goal for unit and integration testing. Each class and public method shall have a JUnit test harness that may be labelled according to their purpose (e.g. I/O, individual class/method tests, software integration, data model integration etc.). Nominal and contingency
tests should be clearly identified.

Interface coverage...

The tool [insert name of unit test coverage tool here] will be used to provide metrics on the code coverage by Unit Tests for \product \ and this metric will be provide in the Test Report.

\subsection{Unit and Integration Test Specification}

This is a example test plan record; this should be generated automatically.

\begin{longtable} {|p{0.2\textwidth}|p{0.2\textwidth}|p{0.6\textwidth}|}\hline
{\bf Class} & {\bf Unit Test Name} & {\bf Purpose}  \\\hline
Unit Test Class & 
Unit Test Method & 
Purpose of Unit Test from method header \\\hline
\end{longtable}


\section{Operations Validation}
 Operations Validation of the system is done through Operations Rehearsals (and or end-to-end tests).
 This may repeat some or all of a science validation exercises but in a more operational setting with a focus on operations. The proposed rehearsal dates are listed in \tabref{tab:ors}.

\begin{longtable} {|l|l|p{0.7\textwidth}|}\hline 
	\caption{Operations rehersals for Ops validaiton of DM \caption{tab:ors}}
{\bf Date/Freq} &{\bf Location}& {\bf Title, Description} \\ \hline

Oct 2018 &  NCSA & {\bf Operations rehearsal for commissioning }
With TBD weeks commissioning (lets say a week) - pick which parts of plan we could rehearse.
Chuck suggests Instrument Signal Removal should be the focus of this (or the next rehearsal).
Oct 2019 & NCSA &  {\bf  Operations rehearsal \#2 for commissioning} 
More complete rehearsal - where do the scientist look at quality data? How do they feed it back to the Telescope ?
Jan 2020 & Base  &  {\bf  Operations rehearsal \#3 for commissioning} 
Dress rehearsal - Just like it will be April for the actual commissioning.
Dec 2020 &  NCSA &  {\bf Operations  Rehearsal Data Release (Commissioning Data)}
2021 &  NCSA &  {\bf Operations  Rehearsal Data Release (Regular Data).}
Feb 2022 &  NCSA/Base &  {\bf Operations  Rehearsal(s)}
Rehearsals for real operations which start Oct 2022
Sept 2022 &  NCSA/Base &  {\bf Operations  Rehearsal(s)}
Full Dress rehearsal for real operations which start Oct 2022

\hline

\end{longtable}




\section{Science Validation} \label{sect:scival}

\subsection{Definition}

We define  \textbf{DM Science Validation} as \textbf{the process by which we
assess the as-built Data Management system meets the needs of the
scientific community and other identified stakeholders}.

We assess the projected and realized scientific usability of the system by
periodically exercising the integrated system in a way that goes beyond
synthetic unit and integration tests and verification of piece-wise
requirements as described in previous sections.  In other words, we {\em
attempt to use the system in ways we expect it to be used by the ultimate
users of the system, scientists}.  An example may be performing a mock
science study on the results of processing of precursor data, or performing
a mock science-like activity (e.g., interactive analysis of time-domain
datasets) on a partially stood-up service (e.g., the Notebook aspect of the
LSST Science Platform).  We record and analyze any issues encountered in
such usage, and feed this information back to the DM Science and DM
development teams.

Science Validation exercises are designed to close the design-build-verify
loop, and enable one to measure the degree to which the requirements,
designs, the as-built system, and future development plans continue to
satisfy stakeholder needs.  They also provide valuable feedback about
modifications needed to ensure the delivery of a scientifically capable
system.  Ultimately, SV activities transfer into commissioning SV activities
and provide training to the future members of the Commissioning team.

% While this Section describes the high-level goals, scope, and organization
% of the Science Verification effort, more details are provided in
% Document-XXX.

\subsection{Schedule and Execution}

\subsubsection{Schedule}

DM SV activities are planned and prepared in a rolling wave fashion in
parallel with development activities (on a 6-month cycle, or perhaps a
year). The SV activities will typically be designed so as to exercise the
capabilities of the system expected to be delivered at the end of a given
development cycle. These follow a long-term roadmap of SV activities,
linked to product delivery milestones in the DM's Construction Plan (see
the table in Section~\ref{sect:schedule}).  The Science Validation (SV) team
guides the definition of goals of those activities, in close consultation
with the DM Project Manager.

By their nature, SV activities will typically lag behind
deliveries of the (sub)system being verified -- ideally, they will commence
immediately upon delivery. Preparatory SV activities (e.g., identification and
acquisition of suitable datasets, identification of potential Science
Collaboration resources to include on the activity, or development of
activity-specific analysis codes) will commence as early as feasible. DM SV
Scientist will coordinate the execution of all SV activities.

SV activities should aim to take no longer than two months to conclude, to
enable rapid actionable feedback to DM Management and DM Subsystem Science.

\subsubsection{Execution}

Science Validation activities typically follow the successful execution of
unit and integration test activities described in the previous sections,
especially the larger ``dress rehearsals'' and ``data challenges'' as
listed in Section~\ref{sect:schedule} (Master Schedule).

Following successful service stand-up or data challenge execution (at
integration and unit test level), the generated data products or integrated
services are turned over to the SV team.  The SV team performs additional
tests and data analyses to exercise the integrated system and assess its
quality relative to expectations for the current phase of construction.
This assessment is fed back to DM Subsystem Science and Systems Engineering
teams to inform them about the status and needed improvements to the system.

Beyond reporting on the results, the SV team examines the tests or
procedures developed in this phase and identifies those that are good new
metrics of system quality and could be run in an automated fashon.  These
are fed back to the development teams for productizing and incorporation
into the automated QC systems.

\subsection{Deliverables}

Key deliverables of Science Validation activities are:
\begin{itemize}

\item Reports on the assessed capability of the Data Management System to
satisfy stakeholder needs.  The assessments shall take into account the
expected maturity of the system being tested.

\item Recommendations for improvements and changes, both in the quality of
as-constructed systems (i.e., what needs to be built differently or better,
to make it more consistent with the system vision), as well as the overall
system vision (i.e., recommendations on where the vision may need to be
modified to fully respond to stakeholder needs).

\item Measurements of performance metrics that do not lend themselves to
easy automation (e.g., science activities requiring human involvement, like
visual classification, or UX tests).

\item Identification of new performance metrics to be tracked, including
potential deliveries of code to the DM Construction and I\&T teams for
inclusion in automated quality control pipelines.

\item Other deliverables as charged when chartering a particular SV exercise.

\end{itemize}

\subsection{Organization and Resources}

\begin{figure}
\centering
\scalebox{0.5}{\hskip 0.0in \includegraphics[trim={2.4cm 3cm 2.4cm
3cm},clip,page=2]{figures/dm-subsystem-science.pdf}}
\caption{Organogram of the Data Management Science Validation Team.
The group is chaired by the DM Science Validation Scientist,
with the DM Science Pipelines Scientist and Institutional Science Leads making up
the permanent membership. Depending on the SV activities being executed at any
given time, the group may draw on additional temporary members from DM SST Staff,
the broader DM Construction staff, as well as external scientists (e.g.,
Science Collaboration members committed to assisting SV goals). SV membership
is reassessed on a cycle by cycle basis, with estimates incorporated in the
long-term plan.
\label{fig:DMsvg}}
\end{figure}

The DM Subsystem Scientist is accountable to the LSST Project Scientist for
successful execution of DM Science Validation activities.  This
responsibility is delegated to the \textbf{DM Science Validation Scientist},
who leads the Science Validation (SV) team.

The SV team guides the definition of goals and receives the products of
dress rehearsal activities, consistent with the long-term testing roadmap
defined in Section~\ref{sect:schedule}.  Decisions on strategic goals of SV exercises are made
in close consultation and coordination with the DM Project Manager and
Subsystem Scientist.  The results of SV activities are reported to the DM
Project Manager and Subsystem Scientist.

SV activities draw on resources of the DM System Science Team, but may also
tap into the broader construction team if needed (and as jointly agreed upon
with the DM Project Manager), as well as contributors from the LSST Science
Collaborations.  Additional members may added as needed, depending on SV
activities being considered and based on the recommendation of the DM SV
Scientist and resource constraints.

The SV Scientist, the DM Science Pipelines Scientist, and all Institutional
Science Leads are ex-officio members of the SV Team.  DM Project Scientist and
Managers are not formal members, but monitor the work of the group.

\subsubsection{Example}

An example of a Science Validation activity may be as follows:

\begin{itemize}

\item Based on the long-term development roadmap and new capabilities
expected to be delivered, the at the beginning of a 6-month cycle the SV
Team defines the goals of a data challenge to be executed at the end of the
cycle.  For the purposes of this example, we assume a major new feature to
be delivered is astrometric calibration and estimation of proper motions.

\item A small data release production using HSC data is defined that
should result in a data set sufficient to measure the size and orientation
of velocity ellipsoids in the Galactic halo.  If such measurement are a
success, they would independently validate the newly added global
astrometric calibration and proper motion measurement capability.

\item At the end the development cycle, the Science Pipelines team delivers to the
proto-Operations team a documented and internally tested set of DRP
pipelines with the new capabilities as defined above.  The pipelines pass
all unit and small-scale integration tests.  The proto-Operations team
deploys and re-verifies the received pipelines in the I\&T environment
designed to closely mimic the production environment.  They verify that the
pipeline integrates well with the orchestration system and is capable of
executing medium-to-large scale processing.  The pipelines pass integration
tests.

\item The data challenge is operationally planned and executed by the
proto-Operations team, including the execution of any predefined QA metrics.
The data products and test results are turned over to the Science
Validation team.

\item The Science Validation team performs the analysis needed to achieve
SV exercise goals (the measurement of vellocity ellipsoids, in this case).

\item The results and conclusions derived from the data challenge are fed back to
the DRP team, DM Project Management, and DM Subsystem Science; they may be
used to assess the overall quality of the product, pass a formal
requirement, and/or inform future construction decisions.

\item Any newly developed but broadly useful tests are identified as such,
and fed to the I\&T team for inclusion into the battery of tests that are
run on a regular basis.

\end{itemize}

