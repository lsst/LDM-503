\section{Software Tools}
\label{sect:tools}

All of the preceding aspects of testing --- verification, science valdation, and operations rehearsals --- rely on a robust and powerful software infrastructure.
In this section, we briefly outline the tooling and practices that are in use
within \product{} to ensure software quality and to verify that requirements are met.
These tools are used continuously (e.g.\ to measure key performance metrics routinely) or periodically (e.g.\ software release characterizations).

\subsection{Continuous Integration and Unit Testing}

Per the \href{https://developer.lsst.io/coding/unit-test-policy.html}{Software Unit Test Policy} described in the \href{https://developer.lsst.io}{DM Developer guide}, all DM code must be accompanied by an appropriate unit test suite.
This suite is regularly executed by a continuous integration (CI) service\footnote{Currently based on Jenkins; \url{https://jenkins.io}}, which is available for on-demand use by developers and for periodic testing (\S\S \ref{sect:ldm-503-nly} \& \ref{sect:ldm-503-wly}).
Irrespective of formally supported platforms, we have a practice of verifying that the codebase can run on at least two distinct operating systems/platforms as portability is often a good proxy for maintainability.

Roles and responsibilities in this area include:

\begin{itemize_single}

\item{The DM Software Architect is responsible for setting the unit testing policy.}

\item{The SQuaRE team is responsible for developing, operating and supporting continuous integration services.}

\item{The SQuaRE team determines platform release practice in conjunction with the other teams, notably including Architecture.}

\end{itemize_single}

At the time of writing, we can calculate but do not formally track the code coverage of our unit test suite.
A system to track this metric will be forthcoming.

\subsection{Code Reviews}

DM’s process requires that every code change is subjected to peer review prior to being merged to the \texttt{master} branch.
This is both as code quality verification and also to ensure that at least one other team-member has some familiarity with a particular part of the codebase.
The code review process is described in the \href{https://developer.lsst.io/work/flow.html#review-preparation}{DM Developer Guide}.

Roles and responsibilities in this area include:

\begin{itemize_single}

\item{The DM Systems Engineering Team defines the development process and style guide including the code review standard.}

\item{SQuaRE is responsible for supporting tooling to assist code review (e.g.\ linters, Jira-GitHub integration, etc).}

\end{itemize_single}

\subsection{Automated Requirements Verification and Metric calculation}

DM uses a harness for continuous metric verification.
In the software development context this is used for:

\begin{itemize_single}

\item Calculating performance metrics and alerting when they fail to meet requirements.

\item Regression testing framework for any developer-supplied metric, with optional alerts when excursions occur from past values to verify that performance is not being degraded by new code or environments.

\item Visualizing these results and linking them back to build and pull request information.

\item Drill-down of those metrics in pre-defined visualization templates geared towards specific verification use-cases.

\end{itemize_single}

Roles and responsibilities in this area include:

\begin{itemize_single}

\item The pipeline teams are responsible for providing the code and data to demonstrate compliance with requirements.

\item SQuaRE is responsible for developing and operating the continuous metric verification services.

\item Individual developers may contribute additional metrics as desired.

\end{itemize_single}
