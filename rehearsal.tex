\section{Operations Rehearsals}

The operability of the \product{} system is demonstrated through a series of operations rehearsals.
Like verification tests , these rehearsals correspond to high level DM milestones (\secref{sect:schedule}), and involve carrying out a specific set of activities under controlled conditions.
As such, many of the considerations described in \secref{sect:approach} also apply to rehearsals.
However, the aim of the rehearsal is not to verify that the performance of the \product{} system meets some requirement, but to verify that it can be integrated and operated successfully, and to demonstrate and validate operational procedures.

The current schedule calls for six rehearsals to be carried out to test different aspects of the system (for example, one rehearsal addresses nightly operations, and another the production and curation of a data release).
At time of writing, the activities to be undertaken as part of each operations rehearsal are currently being detailed.
This schedule and scope of each exercise will be designed to align with the LSST Commissioning Plan (\citeds{lse-79}).

Operations rehearsals require an \emph{Rehearsal Coordinator} to oversee the process.
This is a distinct role from that of the testers (\secref{sect:roles}, since they are (by definition) carrying out their operational roles during the rehearsal.
For example, the rehearsal may not be directed by the Operations Manager, since that person has a major role in the rehearsal.
An individual not involved in the rehearsal itself will be identified to perform this function.
